\section{Conclusion}
\label{sec:conclusion}

%\subsection{Free-Form Visualization}
%- TODO\\
%- Problem studied by hundreds of people - a lot of open sources\\
%- Every step can be visualized\\
%- SOme final visualization



\subsection{Reflection}
As explained in the Section \ref{sec:definition}, try to predict the future stock prices is a known problem and there is people trying to make money out of it. And, when there is people 
studying something, open tools and sources are released. Looking on the internet, you can easily find examples of codes as open source. Also, this code is becoming open source.
With the model defined in this project, it is possible to easily visualize every step of the process, and, as explained in Subsection \ref{subsec:Exploratory_visu}, the code is completely modular.
Implementing new features and visualizations is really fast and easy to fit.\\
\\
Studying for this project helped to understand better about regression methods, how to optimize them and how to preprocess data to best fit them. The most difficult part was initially to understand
how to optimize and visualize all the data available. Also, implementing everything from the beginning was challenging, mostly finding the best way to modularize the code and to implement it with 
an interface. After visualizing everything and enabling to change parameters without hard-coding it the project took a cleaner form, and everything could be organized.\\
\\
This application can run in any stock market and any company on it. Analyzing a company in a short-term, such as in a month, the predictions are very loyal to the real results. However, as stated on
the beginning of this project, this code is not ready to predict a break on the market. For example, if a company is found to be in a corruption scheme, and loses market price, this algorithm will 
wrongly predict all the results until everything normalizes. Therefore, this application should only be used to understand better the stock prices movement. 

\subsection{Improvement}
There is some improvement space to be done is this application. Starting with the prediction algorithm, there is some good results on the internet with deep learning algorithms. There is a sample
of it already implemented, but it is not working fine yet. Using deep learning to implement this kind of prediction would be really interesting. Also, the front end solution presented can be
improved with new frameworks, such as Angular 2. With a solution as this, the website would be more interactive with the user and the visualization could become better and cleaner.\\
\\
However, the most important improvement that should be made, it is to understand the market sentiment in order to predict companies breakdowns and rises. This sentiment could be studied by the movement
of the process, as explained in subsection \ref{subsec:data_exploration}. Also, performing data mining in websites, such as Twitter, can be a very powerful tool to define the sentiment. Since the 
code is modular, introducing this kind of feature on the database and in the prediction algorithm could be done, once this feature is defined and achieved. These are the improvements that are intended
to be done on this code after this part is done.